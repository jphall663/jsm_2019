\documentclass[11pt,
			   %10pt, 
               %hyperref={colorlinks},
               aspectratio=43,
               hyperref={colorlinks}
               ]{beamer}
\usetheme{Singapore}
\usecolortheme[snowy, cautious]{owl}

\usepackage[utf8]{inputenc}
\usepackage[T1]{fontenc}
\usepackage[american]{babel}
\usepackage{graphicx}
\usepackage{hyperref}

\hypersetup{
    colorlinks=true,
    urlcolor=[rgb]{1,0,1},
    linkcolor=[rgb]{1,0,1}}
\definecolor{magenta}{RGB}{255, 0, 255}

\usepackage[natbib=true,style=numeric,backend=bibtex,useprefix=true]{biblatex}

\definecolor{OwlGreen}{RGB}{75,0,130} % easier to see
\setbeamertemplate{bibliography item}{\insertbiblabel}
\setbeamerfont{caption}{size=\footnotesize}
\setbeamertemplate{frametitle continuation}{}

\setcounter{tocdepth}{2}
\renewcommand*{\bibfont}{\scriptsize}
\addbibresource{bibliography.bib}

\renewcommand*{\thefootnote}{\fnsymbol{footnote}}

\usenavigationsymbolstemplate{}
\setbeamertemplate{footline}{%
    \raisebox{5pt}{\makebox{\hfill\makebox[20pt]{\color{gray}
          \scriptsize\insertframenumber}}}\hspace*{5pt}}

\author{\copyright\hspace{1pt}Patrick Hall\footnote{\tiny{This material is shared under a \href{https://creativecommons.org/licenses/by/4.0/deed.ast}{CC By 4.0 license} which allows for editing and redistribution, even for commercial purposes. However, any derivative work should attribute the author and H2O.ai.}}}
\title{Increasing Trust and Understanding in Machine Learning with Model Debugging }
\logo{\includegraphics[height=8pt]{img/h2o_logo.png}}
\institute{\href{https://www.h2o.ai}{H\textsubscript{2}O.ai}}
\date{\today}
\subject{}

\begin{document}
	
	\maketitle
	
	\begin{frame}
	
		\frametitle{Contents}
		
		\tableofcontents{}
		
	\end{frame}

%-------------------------------------------------------------------------------
	\section{What?}
%-------------------------------------------------------------------------------

	\begin{frame}
		
		\frametitle{What is Model Debugging?}
		
		\begin{itemize}
			\item Model debugging is an emergent discipline focused on discovering and remediating errors in the internal mechanisms and outputs of machine learning models.\footnote{See \url{https://debug-ml-iclr2019.github.io/} for numerous model debugging approaches.} 
			\item Model debugging attempts to test machine learning models like code (because the models are code).
			\item Model debugging promotes trust directly and enhances interpretability as a side-effect.
		\end{itemize}
		
	\end{frame}

%-------------------------------------------------------------------------------
	\section{Why?}
%-------------------------------------------------------------------------------

%-------------------------------------------------------------------------------
		\subsection{Inaccuracy}
%-------------------------------------------------------------------------------

			\begin{frame}
		
				\frametitle{Why Bother With Model Debugging?}
		
					\footnotesize{Machine learning models can be \textbf{inaccurate}.}
					\begin{columns}
				
						\column{0.5\linewidth}
						\centering
						\includegraphics[height=125pt]{img/global_shap.png}\\
						\vspace{5pt}
						\tiny{This probability of default classifier, $g_{\text{mono}}$, over-emphasizes the most important feature, a customer's most recent repayment status, $\text{PAY\_0}$.}
				
						\column{0.5\linewidth}
						\centering
						\vspace{9pt}
						\includegraphics[height=120pt]{img/resid.png}\\
						\vspace{8pt}
						\tiny{$g_{\text{mono}}$ also struggles to predict default for favorable statuses, $-2  \leq \texttt{PAY\_0}  < 2$, and often cannot predict on-time payment when recent payments are late, $\text{PAY\_0} \geq 2$}.
				
					\end{columns}
					\normalsize
			
			\end{frame}
	
%-------------------------------------------------------------------------------
		\subsection{Sociological Biases}
%-------------------------------------------------------------------------------	
	
			\begin{frame}
		
				\frametitle{Why Bother With Model Debugging?}
		
				\footnotesize{Machine learning models can perpetuate \textbf{sociological biases} \cite{barocas-hardt-narayanan}.}
				\vspace{10pt}	
				\begin{table}[htb!]
					\centering
					\footnotesize
					\begin{tabular}{ | p{1.2cm} | p{1.1cm} | p{1.3cm} | p{1.2cm}| p{1.2cm} | p{1.2cm} | p{1.2cm} | p{1.2cm} | }
						\hline
						& Adverse\newline Impact\newline Ratio & Accuracy Disparity & TPR\newline Disparity & TNR\newline Disparity & FPR\newline Disparity & FNR\newline Disparity \\ 
						\hline
						\texttt{single} & 0.89 & 1.03 & 0.99 & 1.03 & 0.85 & 1.01 \\
						\hline	
						\texttt{divorced} & 1.01 & 0.93 & 0.81 & 0.96 & \textcolor{magenta}{1.25} & 1.22 \\
						\hline
						\texttt{other} & \textcolor{magenta}{0.26} & 1.12 & \textcolor{magenta}{0.62} & 1.17 & \textcolor{magenta}{0} & \textcolor{magenta}{1.44} \\
						\hline	
					\end{tabular}
				\end{table}
				\footnotesize{Group disparity metrics are out-of-range for $g_{\text{mono}}$ across different marital statuses.}
				\normalsize
		
			\end{frame}
	
%-------------------------------------------------------------------------------
		\subsection{Security Vulnerabilities}
%-------------------------------------------------------------------------------	
	
			\begin{frame}
		
				\frametitle{Why Bother With Model Debugging?}
		
				\footnotesize{Machine learning models can have \textbf{security vulnerabilities} \cite{security_of_ml}, \cite{membership_inference}, \cite{model_stealing}}.
				\begin{figure}[htb]
					\begin{center}
						\includegraphics[height=155pt]{img/cheatsheet.png}
					\end{center}
				\end{figure}	
				\vspace{-17pt}
				\footnotesize{Hackers, competitors, or malicious or extorted insiders can manipulate model outcomes, steal models, and steal data!}
				\normalsize
		
			\end{frame}

%-------------------------------------------------------------------------------
	\section{How?}
%-------------------------------------------------------------------------------

%-------------------------------------------------------------------------------
		\subsection{Holistic, Low-Risk Approach}
%-------------------------------------------------------------------------------	
	
			\begin{frame}
		
				\frametitle{How to Debug Models?}
		
				\footnotesize{As part of a holistic, low-risk approach to machine learning}.\footnote{See \url{https://github.com/jphall663/hc_ml} for more information.}
				\begin{figure}[htb]
					\begin{center}
						\includegraphics[height=170pt]{img/blueprint.png}
					\end{center}
				\end{figure}	
				\normalsize
		
			\end{frame}

%-------------------------------------------------------------------------------
		\subsection{Sensitivity Analysis}
%-------------------------------------------------------------------------------	

			\begin{frame}[allowframebreaks]
		
				\frametitle{\textbf{Sensitivity Analysis}: Search for Adversarial Examples}
		
				\begin{figure}[htb]
					\begin{center}
						\includegraphics[height=170pt]{img/sa_max_prob.png}
					\end{center}
				\end{figure}	
				
				\framebreak
		
				\begin{figure}[htb]
					\begin{center}
						\includegraphics[height=170pt]{img/sa_max_prob_demo.png}
					\end{center}
				\end{figure}		
		
			\end{frame}
			
			\begin{frame}
		
				\frametitle{\textbf{Sensitivity Analysis}: Partial Dependence and Individual Conditional Expectation (ICE)}
		
				\begin{figure}[htb]
					\begin{center}
						\includegraphics[height=83pt]{img/pd.png}
					\end{center}
				\end{figure}	
		
			\end{frame}			
			
			\begin{frame}
		
				\frametitle{\textbf{Sensitivity Analysis}: Random Attacks}
		
				\begin{figure}[htb]
					\begin{center}
						\includegraphics[height=180pt]{img/ra.png}
					\end{center}
				\end{figure}	
		
			\end{frame}
			
%-------------------------------------------------------------------------------
		\subsection{Residual Analysis}
%-------------------------------------------------------------------------------	

			\begin{frame}
		
				\frametitle{\textbf{Residual Analysis}: Disparate Errors}
		
				\begin{figure}[htb]
					\begin{center}
						\includegraphics[height=150pt]{img/de.png}
					\end{center}
				\end{figure}	
		
			\end{frame}

			\begin{frame}
		
				\frametitle{\textbf{Residual Analysis}: Mean Local Feature Contributions}
		
				\begin{figure}[htb]
					\begin{center}
						\includegraphics[height=122pt]{img/global_high_low.png}
					\end{center}
				\end{figure}	
		
			\end{frame}

			\begin{frame}
		
				\frametitle{\textbf{Residual Analysis}: Global Importance for Predictions and Logloss}
		
				\begin{figure}[htb]
					\begin{center}
						\includegraphics[height=150pt]{img/global_pred_loss.png}
					\end{center}
				\end{figure}	
		
			\end{frame}

			\begin{frame}
		
				\frametitle{\textbf{Residual Analysis}: Local Feature Contributions to Logloss}
		
				\begin{figure}[htb]
					\begin{center}
						\includegraphics[height=150pt]{img/local.png}
					\end{center}
				\end{figure}	
		
			\end{frame}

			\begin{frame}
		
				\frametitle{\textbf{Residual Analysis}: Surrogate Decision Trees}
		
				\begin{figure}[htb]
					\begin{center}
						\includegraphics[height=90pt]{img/surrogate_dt_1.png}
					\end{center}
				\end{figure}	
		
			\end{frame}
			
%-------------------------------------------------------------------------------
		\subsection{Benchmark Models}
%-------------------------------------------------------------------------------	

			\begin{frame}
		
				\frametitle{\textbf{Benchmark Models}}
		
				\begin{figure}[htb]
					\begin{center}
						\includegraphics[height=150pt]{img/benchmark.png}
					\end{center}
				\end{figure}	
		
			\end{frame}

%-------------------------------------------------------------------------------
		\subsection{Remediation}
%-------------------------------------------------------------------------------	

%-------------------------------------------------------------------------------
%	\section{References}
%-------------------------------------------------------------------------------

	\begin{frame}[t, allowframebreaks]
	
		\frametitle{References}	
		
			This presentation:\\
					
		\framebreak		
		
		\printbibliography
		
	\end{frame}

\end{document}